\documentclass[11pt]{article}
\usepackage{geometry}
\usepackage{amsmath}
\usepackage{amssymb}
\usepackage{enumitem}
\usepackage{fancyhdr}
\usepackage{tikz}
\usepackage{tcolorbox}
\usetikzlibrary{trees}
\pagestyle{fancy}

\lhead{MTH 201 (Calculus)}
\chead{Activities: Fundamental Theorem of Calculus}
\rhead{2018-11-28}

\begin{document}

\noindent
In this activity, you'll get practice using the \textbf{Fundamental Theorem of Calculus} to compute the exact value of a definite integral. First, let's review what the Fundamental Theorem tells us. Suppose we are given the integral 
$$\int_a^b f(x) \, dx$$
Then to compute its value exactly, we do two things: 

\begin{enumerate}
    \item Find an antiderivative, $F$, for the function $f$ -- that is, a function whose derivative is the function in the integral. 
    \item Then use the antiderivative $F$ and evaluate $F(b) - F(a)$. the result of that subtraction is the value of the integral. 
\end{enumerate}

\begin{tcolorbox}
\textbf{Example:} Use the Fundamental Theorem of Calculus to evaluate $\displaystyle{\int_2^5 (x^2 + 1) \,dx}$.

\noindent
\textbf{Solution:} An antiderivative for $x^2 + 1$ is 
$$F(x) = \frac{1}{3}x^3 + x$$
because the derivative of that function is $x^2 + 1$. Now evaluate: 
$$F(5) - F(2) = \left( \frac{1}{3} \cdot 5^3 + 5 \right) - \left( \frac{1}{3} \cdot 2^3 + 2 \right)
= \left(\frac{125}{3} + 5 \right) - \left( \frac{8}{3} + 2 \right) = \frac{117}{3} + 3 = 42.$$

\end{tcolorbox}

Now work through the following. Use the Fundamental Theorem of Calculus to find the exact value of each integral. Your group will be given one or more of these to work at the board. 

\begin{enumerate}
    \item $\displaystyle{\int_0^1 (x^3 - x - e^x + 2) \, dx}$
    \item $\displaystyle{\int_0^{\pi/3} (2 \sin(t) - 4 \cos(t) + \sec^2 (t) - \pi) \, dt}$
    \item $\displaystyle{\int_0^1 (\sqrt{x} - x^2) \, dx}$
    \item $\displaystyle{\int_1^2 \frac{1}{x} \, dx}$  (Think of a function whose derivative is $1/x$.) 
    \item $\displaystyle{\int_1^{27} \frac{t+1}{\sqrt{t}}\, dt}$ (Simplify first.) 
\end{enumerate}

\begin{tcolorbox}
Answers: $11/4 - e$; $1 - \sqrt{3} - \pi^2/3$; $1/3$; $\ln(2)$; $60\sqrt{3} - 8/3$. 
\end{tcolorbox}

\end{document}