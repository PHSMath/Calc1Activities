\documentclass[11pt]{article}
\usepackage{geometry}
\usepackage{amsmath}
\usepackage{amssymb}
\usepackage{enumitem}
\usepackage{fancyhdr}
\usepackage{palatino}
\usepackage{tikz}
\usepackage{tcolorbox}
\usepackage{url}

\usetikzlibrary{trees}
\pagestyle{fancy}

\lhead{MTH 201}
\chead{Online Activity}
\rhead{2018-10-30}

\begin{document}

\begin{center}
    \Large{Online Module: More Global Optimization}
\end{center}

\noindent
In this activity, you'll get more practice working with the global optimization techniques we learned in Section 3.3, where we find the absolute maximum and minimum values of a continuous function on a closed interval using a three-step process. 

\subsection*{Prerequisites}

Before you can begin this module, you will need to review Guided Inquiry 3.3 and our work from class on Monday October 29. It also helps to work through WeBWorK 16 which focuses on this topic as well, and it's due tonight (October 30). 

\subsection*{Instructions}

In the next section are three problems to work. Please do these in a separate writeup, either neatly hand-written or typed up. Then submit your work according to the instructions at the end of this handout. 


\subsection*{Problems to Work}
    
\begin{enumerate}

\item Find the absolute minimum and absolute maximum values of $f(x) = x^5 - 3x^2$ on the interval $[-1,5]$. This means find both the $x$-coordinate of these points and also state the value of the function at these points. (Example: ``The absolute minimum value of the function is \underline{\hspace{0.2in}} and it occurs at $x=$ \underline{\hspace{0.2in}}.'')

\item Find the absolute minimum and absolute maximum values of $g(x) = 3e^x - e^{2x}$ on the interval $[-1/2,1]$. Give the \emph{exact} values, not decimal approximations. 

\item Go back to the in-class group activity from Monday and work the final problem there, which states: 

\begin{quote}
    The homeowner has 40 feet of fencing to use, and she would like to build the garden with the largest possible area out of that fencing. What should be the length and width of the garden that would create the largest possible area? 
\end{quote}

Follow all the steps on that problem and show all work neatly. 

\end{enumerate}

\vspace{0.5in}

\subsection*{Submission instructions}

Do the work on the problems for this module in your notes, either neatly handwritten or typed. Then submit a PDF of your work (either a PDF version of typed work, or a legible scanned PDF of handwritten work) to the assignment dropbox found on Blackboard in \textbf{Class Materials > Week 10 materials}. This assignment is due by 11:59pm EDT on \textbf{Friday, November 2}. 

Each successfully written solution to the three problems above will earn you 3 XP, so you can earn 9 XP from this activity. 

\end{document}