\documentclass[11pt]{article}
\usepackage{geometry}
\usepackage{amsmath}
\usepackage{amssymb}
\usepackage{enumitem}
\usepackage{fancyhdr}
\usepackage{palatino}
\usepackage{tikz}
\usepackage{tcolorbox}
\usepackage{url}

\usetikzlibrary{trees}
\pagestyle{fancy}

\lhead{MTH 201}
\chead{Online Activity}
\rhead{2018-10-23}

\begin{document}

\begin{center}
    \Large{Online Module: Finding local function behavior using the First and Second Derivative Tests}
\end{center}

\noindent
In this activity, you'll get experience with determining local behavior of functions using the First Derivative Test (for finding intervals of increase/decrease and classifying critical numbers) and the Second Derivative Test (for finding intervals of concavity and inflection points). 

\subsection*{Prerequisites}

Before you can begin this module, you need to complete Guided Inquiry 3.1, which introduces the basic concepts. This was turned in on Sunday, October 21. Please go review the learning resources in that assignment, especially the videos on the First and Second Derivative tests. 

\subsection*{Instructions}

In the next section are four problems to work. Please do these in a separate writeup, either neatly hand-written or typed up. Then submit your work according to the instructions at the end of this handout. 


\subsection*{Problems to Work}
    
\begin{enumerate}

\item A function $f$ has first and second derivatives given by: 
$$f'(x) =5 x^{4} + 10 x^{3} - 15 x^{2}\qquad f''(x) = 20 x^{3} + 30 x^{2} - 30 x$$
    \begin{enumerate}
        \item Factor both of these expressions. Use the factorization of $f'$ to find all the critical numbers of $f$. 
        \item Make a sign chart for $f'$, as explained in the textbook and videos for this section. Give a one-paragraph explanation that describes in detail what you did to construct the chart. Include both the chart and the explanation in your writeup. 
        \item Classify each critical number as a local maximum, local minimum, or neither and explain your reasoning on each. 
        \item Now make a sign chart for $f''$ and use it to determine the intervals on which $f$ is concave up and on which $f$ is concave down. 
        \item State the inflection points of this function and explain your reasoning. 
    \end{enumerate}
    
\item Consider the function $g(x) = \sin(x) + \cos(x)$ on the interval $[0,3]$. 
    \begin{enumerate}
        \item Find $g'$ and $g''$. 
        \item Find all the critical numbers of $g$ on this interval $[0,3]$. Explain your reasoning. 
        \item Make a sign chart for $g'$ and use it to classify each critical number as a local maximum, local minimum, or neither.  Explain your reasoning on each.  
        \item Make a sign chart for $g''$ and use it to determine the intervals on which $f$ is concave up and on which $f$ is concave down. Then Sstate the inflection points of this function on the interval, and explain your reasoning. 
    \end{enumerate}

\item \emph{(A cautionary example about inflection points.)} Let $y = x^4$. 
    \begin{enumerate}
        \item Make a graph of $y$ on Desmos or another tool, and using the graph, answer this question: Does $y$ have any inflection points? 
        \item Find $y''$, then find all the points where $y'' = 0$. 
        \item Are any of the points where $y'' = 0$ inflection points? 
        \item Consider the statement: \emph{Inflection points are places where the function's second derivative is zero.} Is this statement always true? Why or why not? 
    \end{enumerate}

\end{enumerate}

\vspace{0.5in}

\subsection*{Submission instructions}

Do the work on the problems for this module in your notes, either neatly handwritten or typed. Then submit a PDF of your work (either a PDF version of typed work, or a legible scanned PDF of handwritten work) to the assignment dropbox found on Blackboard in \textbf{Class Materials > Week 9 materials}. This assignment is due by 11:59pm EDT on \textbf{Friday, October 26}. 

Each successfully written solution to the three problems above will earn you 3 XP, so you can earn 9 XP from this activity. 

\end{document}