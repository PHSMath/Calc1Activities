\documentclass[11pt]{article}
\usepackage{geometry}
\usepackage{amsmath}
\usepackage{amssymb}
\usepackage{enumitem}
\usepackage{fancyhdr}
\usepackage{palatino}
\usepackage{tikz}
\usepackage{tcolorbox}

\usetikzlibrary{trees}
\pagestyle{fancy}

\lhead{MTH 201}
\chead{Mixed Activity for 2.6, 2.7, 2.8}
\rhead{2018-10-17}

\begin{document}

\begin{center}
    \Large{Mixed Practice --- Derivatives of Inverse Functions, Implicit Differentiation, and Indeterminate Forms}
\end{center}


\subsection*{Derivatives of Inverse Functions}

Find each of the following derivatives. Your group will be given one of these to put on the board. 

\begin{enumerate}
    \item $y = \arctan(t^2 + t+ 1)$
    \item $y = x \ln x$
    \item $y = \ln \left( xe^x \right)$
    \item $y = e^x \arcsin(x)$
    \item $y = \dfrac{1}{\arctan(x)}$
    \item $y = \ln(\ln(x))$
\end{enumerate}


\subsection*{Derivatives of implicit functions}

Consider the equation (not function) $x^3 + y^3 = 1$. Notice that the point $(1,1)$ satisfies this equation. 

\begin{enumerate}
    \item Find $dy/dx$ in terms of $x$ and $y$. Hint: Start by taking the derivative with respect to $x$ of both sides. 
    \item Find the slope of the tangent line to the curve that is traced out by this equation, at the point $(-1,1)$. 
\end{enumerate}

\subsection*{Indeterminate forms and L'Hopital's Rule}

Consider the limit 
$$\lim_{x \to \pi/2} \frac{\cos^2 x}{\sin x - 1}$$

\begin{enumerate}
    \item What do you get when you try to directly evaluate $x = \pi/2$ into the function whose limit you are taking? 
    \item Does the answer to the previous question mean that the limit as $x \to \pi/2$ of this function fails to exist? 
    \item Using Desmos, make a graph of $y = \frac{\cos^2 x}{\sin x - 1}$. Does it appear that the limit as $x \to \pi/2$ exists? If so, what do you think it equals? 
    \item We can use L'Hopital's Rule to evaluate this limit. Why? 
    \item Go ahead and use L'Hopital's Rule to evaluate the limit. 
\end{enumerate}

\end{document}