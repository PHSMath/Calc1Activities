\documentclass[11pt]{article}
\usepackage{geometry}
\usepackage{amsmath}
\usepackage{amssymb}
\usepackage{enumitem}
\usepackage{fancyhdr}
\usepackage{palatino}
\usepackage{tikz}
\usepackage{tcolorbox}

\usetikzlibrary{trees}
\pagestyle{fancy}

\lhead{MTH 201}
\chead{Activities for 3.1 part 2}
\rhead{2018-10-24}

\begin{document}

\begin{center}
    \Large{Activities for finding local extreme values}
\end{center}

Below are six different functions. For each one: 
\begin{itemize}
    \item Find the first and second derivatives. (Check your work using WA or some other tool before moving on!)
    \item Find all the critical numbers of the function.
    \item Make a sign chart for the first derivative and use it to find the intervals on which the function is increasing and decreasing, and then classify each critical number as a local maximum, local minimum, or neither.
    \item Make a sign chart for the second derivative and use it to find the intervals on which the function is concave up and concave down, then identify all the inflection points. 
    \item Check your work by looking at a graph of the original function. 
\end{itemize}

\vspace{0.5in}

\begin{enumerate}
    \item $f(t) = 8t^3 - t^2$
    \item $f(x) = x^3 - \frac{9}{2}x^2 - 54z + 2$
    \item $f(x) = \dfrac{x}{x^2 + 1}$
    \item $f(x) = \dfrac{x^2}{x + 1}$
    \item $f(x) = x \ln x$
    \item $f(t) = (t^2 - 1)^{1/3}$
\end{enumerate}

\end{document}