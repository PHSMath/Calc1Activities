\documentclass[11pt]{article}
\usepackage{geometry}
\usepackage{amsmath}
\usepackage{amssymb}
\usepackage{enumitem}
\usepackage{fancyhdr}
\usepackage{palatino}
\usepackage{tikz}
\usepackage{tcolorbox}

\usetikzlibrary{trees}
\pagestyle{fancy}

\lhead{MTH 201}
\chead{Activities for 3.1}
\rhead{2018-10-17}

\begin{document}

\begin{center}
    \Large{Activities for Local Extreme Values of Functions}
\end{center}


\subsection*{Drawing functions}

Your group will be given a card with one of six sets of properties having to do with local extreme values, critical numbers, and inflection points. Your group will be given one of these. Using the big sticky notes or the whiteboard, draw an example of a graph that has as many of the listed properties as possible. If it's not possible to draw all of these, do as many as you can and then explain why you can't do them all. Be sure to label axes and show scaling. 


\subsection*{Finding and classifying critical numbers}

The derivative of a function $g$ is given by 
$$g'(x) = \dfrac{(x-2)(x+1)}{(x-5)}$$
To repeat: This is the \emph{derivative} of a function $g$, not $g$ itself. (Right now we lack the mathematical tools for reconstructing the formula for $g$.) 

\begin{enumerate}
    \item How many critical numbers does $g$ have, and what are they? 
    \item Think about a value of $x$ that is just to the left of $x = 2$. Is the value of $g'$ at this point positive, negative, or zero?
    \item Think about a value of $x$ that is just to the right of $x = 2$. Is the value of $g'$ at this point positive, negative, or zero?
    \item Does the function $g$ (not its derivative!) have a local minimum at $x=2$, a local maximum, or neither? 
\end{enumerate}

\end{document}