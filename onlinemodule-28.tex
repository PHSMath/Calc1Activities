\documentclass[11pt]{article}
\usepackage{geometry}
\usepackage{amsmath}
\usepackage{amssymb}
\usepackage{enumitem}
\usepackage{fancyhdr}
\usepackage{palatino}
\usepackage{tikz}
\usepackage{tcolorbox}
\usepackage{url}

\usetikzlibrary{trees}
\pagestyle{fancy}

\lhead{MTH 201}
\chead{Online Activity}
\rhead{2018-10-16}

\begin{document}

\begin{center}
    \Large{Online Module: Indeterminate Forms and L'Hopital's Rule}
\end{center}

\noindent
In this activity, you'll gain experience working with indeterminate forms and using L'Hopital's Rule to evaluate indeterminate limits. 

\subsection*{Prerequisites}

Before you can begin this module, you need to complete Guided Inquiry 2.8 which introduces you to indeterminate forms and L'Hopital's Rule. The exercises will help you gain some basic fluency in these concepts that you'll then expand upon in this module. 

\subsection*{Tasks to complete}

\begin{enumerate}
    \item Work through the WeBWorK set ``Online Activity Indeterminate Forms''. This is a three-question set with limits to evaluate. It does not count toward your semester online homework total; for each one you successfully finish, you will earn 1 XP. 
    \item Work through both of the problems in the ``Problems to Work'' section below. You'll submit your work on this through a Google Form. You will earn 3 XP for each problem as long as you submit your work prior to the deadline, all parts of the work are submitted, and the work shows a good-faith effort to be right on each part. 
\end{enumerate}
    
\subsection*{Problems to Work}
    
\begin{enumerate}
    \item Consider the limit 
    $$\lim_{x \to 0} \frac{\sin(x) -x}{\cos(2x)-1}$$
    \begin{enumerate}
        \item Use Desmos to make a graph of the function involved in this limit. Using the graph --- don't use any calculus yet! --- determine whether the limit exists, and if it exists determine what it equals. 
        \item Now try directly substituting $x = 0$ into the function, and make a note of what you get.
        \item Explain why L'Hopital's Rule could be used to evaluate the limit. 
        \item Now use L'Hopital's Rule to evaluate the limit. Write up your work on paper and submit a scanned PDF of your work, or type it up and save as a PDF. Either way, show all your work. 
    \end{enumerate}

    \item Consider the limit 
    $$\lim_{x \to \infty} \frac{e^x + x}{2e^x + x^2}$$
 \begin{enumerate}
        \item Use Desmos to make a graph of the function involved in this limit. Using the graph --- don't use any calculus yet! --- determine whether the limit exists, and if it exists determine what it equals. 
        \item Let $x \to \infty$ on both the top and the bottom, and make a note of what you get. 
        \item Explain why L'Hopital's Rule could be used to evaluate the limit. 
        \item Now use L'Hopital's Rule to evaluate the limit. Write up your work on paper and submit a scanned PDF of your work, or type it up and save as a PDF. Either way, show all your work. 
    \end{enumerate}
\end{enumerate}

\subsection*{Submitting your work}

A Google Form has been set up here: 
\begin{center}
   \url{http://bit.ly/2P1XATj}
\end{center}
There are multiple choice and short-answer items for parts (a)-(c) on each of the problems to work, and a single item asking you to upload electronic versions of part (d) on each problem. You can either upload two files (one for each problem) or one big file that contains work for both problems. 

\vspace{0.5in}

\noindent
\textbf{Reminders about uploading files:}
\begin{enumerate}
    \item You can either hand-write your work on paper or a whiteboard, or type it up. 
    \item If you hand-write your work, please \textbf{scan} the work in using a scanner or a smartphone app that scans work. Some examples of such apps are \textit{GeniusScan}, \textit{Office Lens}, and \textit{ScanBot}. There are many more, almost all of them free. 
    \item However you save your work, \textbf{save your work as a PDF} and not an image (PNG, JPG, etc.) or as a Word file. Please note the Google Form will only accept PDFs. If you are uncertain how to do this, you can Google it, ask a friend, or ask for help on the \#tech channel on Slack. 
    \item Also keep the file you submit to a reasonable size, preferably less than 1 MB. If you are scanning your handwritten work to a PDF, this will be no problem. If you take a photo or scan to an image file, it will be too large. 
\end{enumerate}

\end{document}